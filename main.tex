\documentclass[
    Print       = false,
    Title       = 中国海洋大学硕博学位论文LaTeX模板,
    Author      = 作者,
    Advisor     = 指导教师,
    Thesis      = 全日制专业学位,
    Major       = 专业名称,
    Topic       = 研究方向,
    Year        = 2021,
    Month       = 4,
    Day         = 28,
    TitleEng    = {{LaTeX template for master's and doctoral dissertations of Ocean University of China}}
]{oucthesis}

\begin{document}
    % 封皮
    \makefrontcover
    
    % 致谢
    \makethanks{谨以此文献给敬爱的老师、家人和朋友们!}
    
    % 签名页
    \makesign
    
    % 独创声明
    \makecopyright

    % 中英文摘要
    \begin{abstract}
        中文摘要

        \keywords{关键词1;关键词2;关键词3}
    \end{abstract}

    \begin{enabstract}
        English abstract
        \enkeywords{Keyword 1; Keyword 2; Keyword 3}
    \end{enabstract}
    
    % 目录
    \tableofcontents
    
    \mainmatter

    \chapter{简介}

    \section{一级标题}

    \subsection{二级标题}

    \subsubsection{三级标题}

    \chapter{公式、图与表}

    \section{公式插入}

    一元二次方程如公式(\ref{eq:})所示。

    \begin{equation}
        y=ax^2+bx+c
        \label{eq:}
    \end{equation}

    \section{图片插入}

    中国海洋大学校徽如图 \ref{fig:ouc} 所示。

    \begin{figure}[ht]
        \centering
        \includegraphics[width=.5\textwidth]{img/ouc.jpg}
        \caption{中国海洋大学校徽}
        \label{fig:ouc}
    \end{figure}

    \section{表格插入}

    表 \ref{tab:} 是表格的一个示例。

    \begin{table}[h]
        \centering
        \caption{一元二次方程参数描述}
        \begin{tabular}{cc}
            \toprule
            参数&描述\\
            \midrule
            $a$&二次项系数\\
            $b$&一次项系数\\
            $c$&常数项\\
            \bottomrule
        \end{tabular}
        \label{tab:}
    \end{table}

    \chapter{文献引用}

    TensorFlow\cite{abadi2016tensorflow}是一个基于数据流编程(Dataflow Programming)的符号数学系统,被广泛应用于各类机器学习(Machine Learning)算法的编程实现,其前身是谷歌的神经网络算法库DistBelief。

    % 参考文献
    \makebib{cite}
    
    % 致谢
    \begin{acknowledgement}
        致谢内容
    \end{acknowledgement}

    % 个人简历
    \begin{profile}
        1996年X月X日出生于XX省XX市。

        2014年9月考入XX大学XX专业,2018年6月本科毕业并获得工学学士学位。

        2018年9月考入中国海洋大学XX学院XX专业攻读硕士学位至今。
    \end{profile}

    % 发表的学术论文
    \begin{mypaper}
        \item SAN Z, SI L, WU W. Research on Something that is Important[C]. CVPR 2021: volume 1. IEEE, 2021: I-I
    \end{mypaper}

    % 封皮
    \makebackcover
\end{document}